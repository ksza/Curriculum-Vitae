\documentclass[a4paper,10pt]{article}

%A Few Useful Packages
\usepackage{marvosym}
\usepackage{fontspec} 					%for loading fonts
\usepackage{xunicode,xltxtra,url,parskip} 	%other packages for formatting
\RequirePackage{color,graphicx}
\usepackage[usenames,dvipsnames]{xcolor}
\usepackage[big]{layaureo} 				%better formatting of the A4 page
% an alternative to Layaureo can be ** \usepackage{fullpage} **
\usepackage{supertabular} 				%for Grades
\usepackage{titlesec}					%custom \section
\usepackage{longtable}
\usepackage{array}
%Setup hyperref package, and colours for links
\usepackage{hyperref}
\definecolor{linkcolour}{rgb}{0,0.2,0.6}
\hypersetup{colorlinks,breaklinks,urlcolor=linkcolour, linkcolor=linkcolour}

%FONTS
\defaultfontfeatures{Mapping=tex-text}
\setmainfont[SmallCapsFont = Fontin SmallCaps]{Fontin}

%CV Sections inspired by: 
%http://stefano.italians.nl/archives/26
\titleformat{\section}{\Large\scshape\raggedright}{}{0em}{}[\titlerule]
\titlespacing{\section}{0pt}{3pt}{3pt}
%Tweak a bit the top margin
%\addtolength{\voffset}{-1.3cm}

%Italian hyphenation for the word: ''corporations''
\hyphenation{im-pre-se}

%--------------------SOME CUSTOM COMMANDS----------------------
%\newcommand{\projecttype}[1]{ \raggedleft 
%\footnotesize{\emph{\underline{#1}}}}
%\newcommand{\pSchool}{\projecttype{school}}
%\newcommand{\pOpensource}{\projecttype{opensource}}
%\newcommand{\pIndividual}{\projecttype{individual}}
%\newcommand{\pDiploma}{\projecttype{diploma}}

%-------------WATERMARK TEST [**not part of a CV**]---------------
\usepackage[absolute]{textpos}

\setlength{\TPHorizModule}{30mm}
\setlength{\TPVertModule}{\TPHorizModule}
\textblockorigin{2mm}{0.65\paperheight}
\setlength{\parindent}{0pt}

%--------------------BEGIN DOCUMENT----------------------
\begin{document}

\pagestyle{plain} % non-numbered pages

\font\fb=''[cmr10]'' %for use with \LaTeX command

%--------------------TITLE-------------
 
\par{\centering
		{\Huge \textsc{K\'aroly} Sz\'ant\'o
	}\bigskip\par}
	
%--------------------SECTIONS-----------------------------------
%Section: Personal Data
\section{Personal Data}

\begin{tabular}{rl}
    \textsc{Place of Birth:} & Popesti, Bihor, Romania\\
    \textsc{Date of Birth:}  & 14th of July 1985 \\
%\textsc{Address:}   	 &  Grønjordskollegiet 1, 2708, Copenhagen S, 2300,    
%Denmark\\ 
    \textsc{Address:}         & Strada Eugen Nagy Martir, Nr. 18, Bl. 30, Sc. A, Ap. 20, Timisoara, Romania\\
    \textsc{Phone:}     	 & +40 742 83 10 88\\
    \textsc{Fax:}			 & +40 356 81 82 37\\
    \textsc{email:}     	 &
    \href{mailto:szanto.karoly@gmail.com}{szanto.karoly@gmail.com}
\end{tabular}

%Section: Work Experience at the top
\section{Work Experience}
\begin{longtable}{p{2.5cm}|p{11cm}}

 \raggedleft \textsc{August 2012} & Technical Lead at
 \textsc{\href{http://www.3pillarglobal.com/}{3Pillar Global}}, Timisoara, Romania
 \\\raggedleft \textsc{Present}\\& 
 \footnotesize{A leading technology services provider and software product development company.}\\
& \footnotesize{I am currently working as a technical lead on mobile technologies developing a couple of Android newsreader applications for \textsc{\href{http://www.telegraph.co.uk/}{The Telegraph}}. Before this engagement, I was involved in a short collaboration for an Android project with \textsc{\href{https://www.geico.com/}{Geico}}. My first assignment as a technical lead was a project is a Microsoft SharePoint client implemented both on Android and iOS platforms. The responsibilities I had as a technical lead include: coordinating a team of up to 10 people (dev \& QA), overall architecture of the application, constant communication with the client to make sure we are building both the right product \& the product right, actual task development, etc} \\ \\
& \footnotesize{Previously, I have participated in the development of Spring Data SimpleDB, an in-house opensource project. The Spring Data SimpleDB module aims to provide a familiar and consistent Spring-based programming model for Amazon SimpleDB while retaining domain-specific features and capabilities. Key functional areas of Spring Data SimpleDB are a POJO centric model for interacting with a SimpleDB domains and easily writing a Repository style data access layer.} \\
& \footnotesize{\emph{GitHub (\url{https://github.com/3pillarlabs/spring-data-simpledb})}} \\ \\
& \footnotesize{As a first assignment in the company, I have worked as a Senior Software Developer on a market inteligence project employing several Java technologies in the backend (Hibernate, Google Guice, Apache Shiro) and JavaScript technologies in the
frontend (jQuery, Backbone, Bootstrap). I have been responsible with development of modules both in the backend and the frontend.}\\
\multicolumn{2}{c}{} \\ 

 \raggedleft \textsc{Februay 2012} & Programmer at
 \textsc{\href{http://www.noitso.dk/}{noitso}}, Copenhagen, Denmark
 \\\raggedleft \textsc{July 2012}\\& 
 \footnotesize{A company focused on delivering business-critical IT solutions, helping companies to optimize their workflows.}\\

& \footnotesize{At noitso I worked as a developer focused on C\# and ASP.NET and a few technologies .NET, ADO.NET Entity
Framework, Windows Workflow Foundation}\\
\multicolumn{2}{c}{} \\ 

 \raggedleft \textsc{February 2011} & Programmer at
 \textsc{\href{http://www.itu.dk/pit/}{pIT Lab}}, IT University of Copenhagen
 \\\raggedleft \textsc{July 2012}\\& 
 \footnotesize{The Pervasive Interaction Technology Laboratory hosts research
 projects run by the faculty of the IT University of Copenhagen and
 collaborating institutions.}\\

 & \footnotesize{My main responsibilities in the lab were to fulfill technical
 tasks in the \href{http://www.monarca-project.eu/}{MONARCA Project}, being constantly involved in the  development, maintenance and testing of the \emph{Android client application}, the \emph{infrastructure} (including server management) and of the \emph{web application}. The central infrastructure consists of a server running \href{http://couchdb.apache.org}{couchDB}, \href{http://www.joomla.org} and \href{http://httpd.apache.org}{apache server}. Each user has an Android enabled device running the client application and a mobile version of the couchDB, linked to the couchDB instance running on the central server.}\\

& \footnotesize{Besides the MONARCA Project, I was also involved in ad-hoc programming tasks, as help in other projects was required.}\\
\multicolumn{2}{c}{} \\ 

 \raggedleft \textsc{August 2010} & Software Developer at \textsc{\href{http://www3.oce.com/ro/}{Oc\'e}} srl, Timisoara, Romania
 \\\raggedleft \textsc{May 2007}\\& 
 \footnotesize{Oc\'e, Netherlands based company, is a leading international
 provider of innovative digital printing systems, software
 and services for the production, reproduction, distribution and management of
 documents. Supporting the workflow solutions are Oc\'e digital  printers and
 scanners, considered the most reliable and productive in the world.}\\ \\ 
 &\emph{Tcl/Tk developer}\\
 &\footnotesize{Responsible for the development and maintenance of a tool
 used for automated execution of test cases. The tool and the associated scrips
 are developed in Tcl/Tk. Also part of the responsibilities was the update of
 the project documentation: design papers, interface descriptions and user
 manuals.}\\ 
 &\emph{Java developer}\\&\footnotesize{Part of the team developing the 
 graphical user interface for the touch-screen based Oc\'e printers. Enhanced
 the functionality and refactored existing code. Responsible with the development
 of graphical elements and custom \emph{Swing} components.} 
 \\\multicolumn{2}{c}{} \\ 
\end{longtable}

%Section: Publications
\section{Publications}
\begin{tabular}{p{2.5cm}|p{11cm}}

 \raggedleft \textsc{2013} & Designing mobile health technology for bipolar disorder: \\
 & \emph{a field trial of the monarca system} \\
& \footnotesize{(\href{http://dl.acm.org/citation.cfm?id=2481364&dl=ACM&coll=DL&CFID=386217439&CFTOKEN=60795757}{available on ACM})} \\
& \\
& \footnotesize{Jakob E Bardram, Mads Frost, Károly Szántó, Maria Faurholt-Jepsen, Maj Vinberg and Lars Vedel Kessing. Designing mobile health technology for bipolar disorder: a field trial of the monarca system. In Proceedings of the SIGCHI Conference on Human Factors in Computing Systems. 2013, 2627–2636}\\
 \multicolumn{2}{c}{} \\

 \raggedleft \textsc{2012} & The MONARCA Self-assessment System: \\
 & \emph{a persuasive personal monitoring system for bipolar patients} \\
& \footnotesize{(\href{http://dl.acm.org/citation.cfm?id=2110370}{available on ACM})} \\
&\\
& \footnotesize{Jakob E Bardram, Mads Frost, Károly Szántó and Gabriela Marcu. The MONARCA self-assessment system: a persuasive personal monitoring system for bipolar patients. In Proceedings of the 2nd ACM SIGHIT International Health Informatics Symposium. 2012, 21–30}\\
 \multicolumn{2}{c}{} \\

 \raggedleft \textsc{2011} & The MONARCA Self-assessment System: \\
 & \emph{Persuasive Personal Monitoring for Bipolar Patients} \\
& \footnotesize{(\href{http://ieeexplore.ieee.org/xpl/articleDetails.jsp?tp=&arnumber=6038793&url=http%3A%2F%2Fieeexplore.ieee.org%2Fxpls%2Fabs_all.jsp%3Farnumber%3D6038793}{available on IEEEXplore})} \\
& \\
& \footnotesize{M Frost, G Marcu, R Hansen, K Szaanto and J E Bardram. The MONARCA self-assessment system: Persuasive personal monitoring for bipolar patients. In Pervasive Computing Technologies for Healthcare (PervasiveHealth), 2011 5th International Conference. 2011, 204-205}\\
\end{tabular}

%Section: Education
\section{Education}
\begin{tabular}{p{2.5cm}|p{11cm}}
\raggedleft \textsc{September 2010} & Master of Science in IT, Software Development and Technology
\\ \raggedleft \textsc{August 2014} & IT University of Copenhagen,
København, Denmark \\ \\
& \footnotesize{Specialisation: Pervasive Computing} \\
& \footnotesize{Thesis: \href{http://karolyszanto.ro/MastersThesis/An%20Environment%20Simulator%20for%20Mobile%20Context-Aware%20System%20Design.pdf}{An Environment Simulator for Mobile
Context-Aware System Design}} \\
& \footnotesize{Thesis Project: EgoSim (\url{https://github.com/ksza/EgoSim})} \\
 \multicolumn{2}{c}{} \\
\raggedleft \textsc{2004} & Diploma Engineer in Systems Engineering and
Computers Engineering, \\ \raggedleft \textsc{2009} & ``Politehnica'' University of Timisoara, Romania, College of Automation and
Computers.\\ \\
& \footnotesize{Thesis: \href{http://karolyszanto.ro/Diploma/KarolySzanto_thesis.pdf}{Automation of Complex Design Flaws Detection}. An Eclipse plug-in which detects design
 problems in projects and gives restructuring hints based on design patterns.} \\
\multicolumn{2}{c}{} \\ 
\raggedleft \textsc{2000} & Class of Mathematics-Informatics at the
``Mihai Eminescu'' National \\ \raggedleft \textsc{2004} &
College, Oradea, Romania.\\ \\
& \footnotesize{Project: File Manager Application (\url{https://github.com/ksza/KNavigator}). A clone of the well known DosNavigator file manager} \\
\end{tabular}

%Section: Certifications and Trainings
\section{Certifications and Trainings}
\begin{tabular}{p{2.5cm}|p{11cm}}
\raggedleft \textsc{January 2010} & Sun Certified Programmer for the Java 2
Platform, Standard Edition 6.0. 
\\
\multicolumn{2}{c}{}
\\
 \raggedleft \textsc{2007} & Test and Validation course, by Oc\'e.\\
\multicolumn{2}{c}{} 
\\ 
\raggedleft  \textsc{2005} & Attended and successfully
graduated the first Cisco Certified Network Associate module.\\ \multicolumn{2}{c}{} \\
\end{tabular}

%Section: Undertaken Projects
\section{Undertaken Projects}
\begin{longtable}{p{2.5cm}|p{11cm}}

 \raggedleft \textsc{February 2012} & Realms - Creating Location-Based Services through Configuration\\ 
 \raggedleft \textsc{June 2012} &
 \footnotesize{Paper: (\url{http://itu.dk/people/ksza/reports/realms_project_report.pdf})}\\ &
 \footnotesize{Code:(\url{https://github.com/ksza/itu_realms_project})}\\ &
 \footnotesize{Deployed App:(\url{http://nexgsd.org:8080/realms_configurator})}\\ &
 \footnotesize{An an end-user programming system - Realms - that allows users to augment
physical locations with digital information. With Realms, users can augment physical locations with information through a web-interface centered
around a simple map. The information is thereafter made available on the mobile phone application depending on basic rules and the behavior of the
mobile phone user.}\\
 \multicolumn{2}{c}{}\\

 \raggedleft \textsc{September 2011} & Electronic Announcement Board\\ 
 \raggedleft \textsc{December 2011} &
 \footnotesize{Code:(\url{http://subversion.assembla.com/svn/jsf2_websop/trunk/web_shop_spring})}\\ &
 \footnotesize{Deployed
 App:(\url{http://130.226.142.171:9080/shop_spring})}\\ &
 \footnotesize{Developed using the following Java-based technologies and
 frameworks: JSF 2.1, Hibernate, Spring 3.}\\
 \multicolumn{2}{c}{}\\

 \raggedleft \textsc{February 2011} & "LaPizzeria" webshop\\ 
 \raggedleft \textsc{April 2011} & \footnotesize{(\url{https://bitbucket.org/KarolySzanto/lapizzeria_jsp})}\\ & \footnotesize{(\url{https://bitbucket.org/KarolySzanto/lapizzeria_jsf})}\\
& \footnotesize{A web application developed for didactic reasons, using Java-based technologies and frameworks: Servlet and JSP, JSF and Hibernate. The applications were deployed in Tomcat.}\\
 \multicolumn{2}{c}{} \\

 & Jolie\\
 & \emph{Java Orchestration Language Interpreter Engine}\\
 & \footnotesize{(\url{http://www.jolie-lang.org})}\\
 & \footnotesize{A programming language based upon the service-oriented programming paradigm, suitable to create new services from scratch or to compose existing ones in order to obtain new functionalities.}\\

& \footnotesize{I have contributed to this project implementing a
 few missing operators (+=, -=, /=, *=) and implementing a new communication protocol based on \href{http://json-rpc.org/}{JSON-RPC}.}\\ \multicolumn{2}{c}{} \\

 \raggedleft \textsc{September 2010} & JCAF\\
 \raggedleft \textsc{December 2010} & \emph{The Java Context-Awareness
 Framework} \\ 
 & \footnotesize{Code: (\url{https://sourceforge.net/projects/jcaf})}\\
 &
 \footnotesize{Paper: (\url{http://sourceforge.net/projects/jcaf/files/documentation/RESTful_jcaf_for_Android.pdf})}\\
 & \footnotesize{Context-awareness is a key concept in \href{http://en.wikipedia.org/wiki/Ubiquitous_computing}{ubiquitous~computing}.
 JCAF is a Java-based context-awareness infrastructure and programming API for
 creating context-aware applications on the J2SE platform. I have extended the
 framework to support the development of context-aware applications on the
 Android platform by implementing a communication mechanism based on RESTful
 web services (using \href{http://www.restlet.org/}{Restlet}), as an alternative
 to the existing RMI based communication. I demonstrated my work with an
 Android application written on top of the extended framework.}\\
 \multicolumn{2}{c}{} \\
 \multicolumn{2}{c}{} \\
 \multicolumn{2}{c}{} \\
 \multicolumn{2}{c}{} \\
 \multicolumn{2}{c}{} \\
 
 
 & Bicycle Shop\\
 & \footnotesize{(\url{https://github.com/ksza/Bicycle-Shop})}\\
 & \footnotesize{An IT solution for a fictive bicycle shop enabling users to
 easily create custom bicycles from available parts. Invoking a model driven approach, the application was developed as several Eclipse plugins, making use of EMF, GMF, XText and XPand.}\\ 
 \multicolumn{2}{c}{} \\
  
 & Fluid Photo Browser\\
 & \emph{A multi-device photo browser}\\
 &
 \footnotesize{(\url{https://github.com/ksza/Fluid-Photo-Browser})}\\ 
 & \footnotesize{The application enables users to move photos from
 smart phones, holding personal information, to a shared/public tabletop where
 many people can browse the photos all together}\\ 
 \multicolumn{2}{c}{} \\
 
 \raggedleft \textsc{September 2009} & Kigun \\
 \raggedleft \textsc{April 2010} & \emph{A dental cabinet management project}
 \\ & \footnotesize{The project is built upon the Apache Felix framework (an open source implementation of the OSGi Release 4 core framework specification)
 using the following technologies: Hibernate, Swing, JasperReports} \\ \\
 & \footnotesize{The team has three members (including myself) working from
 different geographical locations, practicing agile development.} \\
 \multicolumn{2}{c}{} \\ 
  
 \raggedleft \textsc{2008} & Mini Java compiler \\ 
 &
 \footnotesize{(\url{https://github.com/ksza/MiniJavaCompiler})}\\
 & \footnotesize{\href{http://compilers.cs.ucla.edu/vids/MCIIJ2E}{Mini Java} is
a subset of Java. The compiler is written Java using the JavaCC parser generator.} \\
 \multicolumn{2}{c}{} \\
  
 \raggedleft \textsc{2007} & Download Accelerator \\
 &
 \footnotesize{(\url{https://github.com/ksza/DownloadAccelerator})}\\
 & \footnotesize{A client-server application written in standard C, under
 Unix. The application offers the possibility to pause and resume a download. The client can download
 from multiple servers simultaneously.}\\ 
 \multicolumn{2}{c}{} \\
 
 & YAGLE\\
 &\footnotesize{(\url{https://github.com/ksza/Yagle})}\\
 & \footnotesize{On-line multiplayer arcade games. Written as a Java
client-server application.}\\ \multicolumn{2}{c}{} \\
 
 \raggedleft \textsc{2005} & Adolix Wallpaper Changer \\
 &
 \footnotesize{(\url{https://github.com/ksza/Adolix-Wallpaper-Changer})}\\
 & \footnotesize{A
 desktop wallpaper manager, written in Delphi
 (\url{http://www.adolix.com/wallpaper-changer}).} \\
 \multicolumn{2}{c}{} \\
\end{longtable}

\section{Technical Skills}
\begin{tabular}{p{2.5cm}p{11cm}}
\textsc{Programming} & Java, C, C++, C\#, Delphi\\
\textsc{Scripting} & javascript, Tcl/Tk, Bash\\ 
\multicolumn{2}{c}{} \\ 
\textsc{Technologies} & XML, Swing, log4j, jUnit, Spring, Hibernate, Restlet, Servlet, JSP, JSF, MySQL, Microsoft SQL Server, couchDB, .NET, ADO.NET Entity Framework, Windows Workflow Foundation\\ 
 \multicolumn{2}{c}{} \\
 \textsc{Operating} & MS-DOS, Windows, Linux/UNIX, Android\\
 \textsc{systems}&\\ 
 \multicolumn{2}{c}{} \\
 \textsc{Other} & Knowledge of Eclipse plug-in development environment and model
 driven development technologies (EMF, GMF, XText, XPand)\\
 & Intermediate knowledge of \LaTeX \\
 & HTML, CSS, jQuery.js, Backbone.js, Bootstrap.js, require.js\\
\end{tabular}

%Section: Languages
\section{Languages}
\begin{tabular}{p{2.5cm}p{11cm}}
\textsc{Romanian}& Mothertongue\\
\textsc{Hungarian}& Fluent\\
\textsc{English}& Fluent\\
\end{tabular}

\section{Abilities and aptitudes}
Ability to work under pressure.
Creativity, initiative and the ability to work both independently and as part of
a team. Good sense of humour.

\section{Interests and Activities}
I enjoy sports, playing different kind of musical instruments (guitar and
didgeridoo), traveling and culinary arts.

\section{About this CV}
This Curriculum Vitae is written in \LaTeX. You can find the source code on GitHub at \url{https://github.com/ksza/Curriculum-Vitae}.

\section{References}
Available upon request.

%\newpage
%\hypertarget{gmat}{\textsc{Gmat}\setmainfont{LMRoman10 Regular}\textregistered\setmainfont[SmallCapsFont=Fontin-SmallCaps]{Fontin-Regular}}

%\XeTeXpdffile ''GMAT.pdf'' page 1 scaled 800

\end{document}
