\documentclass[a4paper,10pt]{article}

%A Few Useful Packages
\usepackage{marvosym}
\usepackage{fontspec} 					%for loading fonts
\usepackage{xunicode,xltxtra,url,parskip} 	%other packages for formatting
\RequirePackage{color,graphicx}
\usepackage[usenames,dvipsnames]{xcolor}
\usepackage[big]{layaureo} 				%better formatting of the A4 page
% an alternative to Layaureo can be ** \usepackage{fullpage} **
\usepackage{supertabular} 				%for Grades
\usepackage{titlesec}					%custom \section
\usepackage{longtable}
\usepackage{array}
%Setup hyperref package, and colours for links
\usepackage{hyperref}
\definecolor{linkcolour}{rgb}{0,0.2,0.6}
\hypersetup{colorlinks,breaklinks,urlcolor=linkcolour, linkcolor=linkcolour}

%FONTS
\defaultfontfeatures{Mapping=tex-text}
\setmainfont[SmallCapsFont = Fontin SmallCaps]{Fontin}

%CV Sections inspired by: 
%http://stefano.italians.nl/archives/26
\titleformat{\section}{\Large\scshape\raggedright}{}{0em}{}[\titlerule]
\titlespacing{\section}{0pt}{3pt}{3pt}
%Tweak a bit the top margin
%\addtolength{\voffset}{-1.3cm}

%Italian hyphenation for the word: ''corporations''
\hyphenation{im-pre-se}

%-------------WATERMARK TEST [**not part of a CV**]---------------
\usepackage[absolute]{textpos}

\setlength{\TPHorizModule}{30mm}
\setlength{\TPVertModule}{\TPHorizModule}
\textblockorigin{2mm}{0.65\paperheight}
\setlength{\parindent}{0pt}

%--------------------BEGIN DOCUMENT----------------------
\begin{document}

\pagestyle{plain} % non-numbered pages

\font\fb=''[cmr10]'' %for use with \LaTeX command

%--------------------TITLE-------------
 
\par{\centering
		{\Huge \textsc{K\'aroly} Sz\'ant\'o
	}\bigskip\par}
	
%--------------------SECTIONS-----------------------------------
%Section: Personal Data
\section{Personal Data}

\begin{tabular}{rl}
    \textsc{Place of Birth:} & Popesti, Bihor, Romania\\
    \textsc{Date of Birth:}  & 14th of July 1985 \\
    \textsc{CPR:} 			 & 140785-3373 \\ \\
    \textsc{Address:}   	 &  Grønjordskollegiet 1, 2708,
    København S, 2300, Denmark \\ 
    \textsc{Phone:}     	 & +45 502 85 221\\
    \textsc{Fax:}			 & +40 356 81 82 37\\
    \textsc{email:}     	 &
    \href{mailto:szanto.karoly@gmail.com}{szanto.karoly@gmail.com}
\end{tabular}

%Section: Work Experience at the top
\section{Work Experience}
\begin{tabular}{p{2.5cm}|p{11cm}}

 \raggedleft \textsc{February 2011} & Programmer at
 \textsc{\href{http://www.itu.dk/pit/}{pIT Lab}}, IT University of Copenhagen
 \\\raggedleft \textsc{July 2011}\\& 
 \footnotesize{The Pervasive Interaction Technology Laboratory hosts research
 projects run by the faculty of the IT University of Copenhagen and
 collaborating institutions.}\\ \\

 & \footnotesize{My main responsibilities in the lab were to fullfil technical tasks in the \href{http://www.monarca-project.eu/}{MONARCA Project}, being constantly involved in the  development, maintainance and testing of the \emph{Android client application}, the \emph{infrastructure} (including server management) and the \emph{web application}. The central infrastructure consists of a server running \href{http://couchdb.apache.org}{couchDB}, \href{http://www.joomla.org} and \href{http://httpd.apache.org}{apache server}. Each user has an Android enabled device running the client application and a mobile version of the couchDB, linked to the couchDB instance running on the central server. Through the Joomla system we offer the users the facility of accessing their data and other features
 }\\\\

 \raggedleft \textsc{August 2010} & Software Developer at \textsc{\href{http://www3.oce.com/ro/}{Oc\'e}} srl, Timisoara, Romania
 \\\raggedleft \textsc{May 2007}\\& 
 \footnotesize{Oc\'e, Netherlands based company, is a leading international
 provider of innovative digital printing systems, software
 and services for the production, reproduction, distribution and management of
 documents. Supporting the workflow solutions are Oc\'e digital  printers and
 scanners, considered the most reliable and productive in the world.}\\ \\ 
 &\emph{Tcl/Tk developer}\\
 &\footnotesize{Responsible for the development and maintenance of a tool
 used for automated execution of test cases. The tool and the associated scrips
 were developed in Tcl/Tk. Also part of the responsibilities was the update of
 the project documentation: design papers, interface descriptions and user
 manuals.}\\ 
 &\emph{Java developer}\\&\footnotesize{Part of the team developing the 
 graphical user interface for the touch-screen based Oc\'e printers. Enhanced
 the functionality and refactored existing code. Responsible with the development
 of graphical elements and custom \emph{Swing} components.} 
 \\\multicolumn{2}{c}{} \\ 
 \raggedleft \textsc{Summer 2006} & Summer Work and Travel, Cape May, New
 Jersey, USA\\&\footnotesize{Kitchen assistant at the \emph{Black Duck on Sunset} restaurant.}
\end{tabular}

%Section: Publications
\section{Publications}
\begin{tabular}{p{2.5cm}|p{11cm}}

 \raggedleft \textsc{May 2011} & The MONARCA Self-assessment System \\
 & Persuasive Personal Monitoring for Bipolar Patients \\
\end{tabular}

%Section: Education
\section{Education}
\begin{tabular}{p{2.5cm}|p{11cm}}
\raggedleft \textsc{September 2010} & M.Sc. in Software Engineering \\
\raggedleft \textsc{June 2011} & IT University of Copenhagen, København,
Denmark \\ \multicolumn{2}{c}{} \\
\raggedleft \textsc{2004} & Diploma Engineer in Systems Engineering and
Computers Engineering, \\ \raggedleft \textsc{2009} & ``Politehnica'' University of Timisoara, Romania, College of Automation and
Computers.
\\\multicolumn{2}{c}{} \\ \raggedleft \textsc{2000} & Class of Mathematics-Informatics at the
``Mihai Eminescu'' National \\ \raggedleft \textsc{2004} &
College, Oradea, Romania.\\
\end{tabular}

%Section: Certifications and Trainings
\section{Certifications and Trainings}
\begin{tabular}{p{2.5cm}|p{11cm}}
\raggedleft \textsc{January 2010} & Sun Certified Programmer for the Java 2
Platform, Standard Edition 6.0. 
\\
\multicolumn{2}{c}{}
\\
 \raggedleft \textsc{2007} & Test and Validation course, by Oc\'e.\\
\multicolumn{2}{c}{} 
\\ 
\raggedleft  \textsc{2005} & Attended and successfully
graduated the first Cisco Certified Network Associate module.\\ \multicolumn{2}{c}{} \\
\end{tabular}

%Section: Education
\section{Undertaken Projects}
\begin{longtable}{p{2.5cm}|p{11cm}}
 \raggedleft \textsc{September 2010} & JCAF\\
 \raggedleft \textsc{December 2010} & \emph{The Java Context-Awareness
 Framework} \\ 
 & \footnotesize{(\url{https://sourceforge.net/projects/jcaf})}\\
 & \footnotesize{Context-awareness is a key concept in
 \href{http://en.wikipedia.org/wiki/Ubiquitous_computing}{ubiquitous~computing}.
 JCAF is a Java-based context-awareness infrastructure and programming API for
 creating context-aware applications on the J2SE platform. I have extended the
 framework to support the development of context-aware applications on the
 Android platform by implementing a communication mechanism based on RESTful
 web services (using \href{http://www.restlet.org/}{Restlet}), as an alternative
 to the existing RMI based communication. I demonstrated my work with an
 Android application written on top of the extended framework.}\\
 \multicolumn{2}{c}{} \\
 
 & Noware Bicycle Shop\\
 & \footnotesize{(\url{http://subversion.assembla.com/svn/mdd2010})}\\
 & \footnotesize{An IT solution for a fictive bicycle shop enabling users to
 easily create custom bicycles from available parts. The application was
 developed, invoking a model driven development approach, as several Eclipse
 plug-ins, making use of the following technologies: EMF, GMF, XText and
 XPand.}\\ 
 \multicolumn{2}{c}{} \\
  
 & Fluid Photo Browser\\
 & \emph{A multi-device photo browser}\\
 &
 \footnotesize{(\url{http://subversion.assembla.com/svn/fluid_photo_browser/trunk})}\\ 
 & \footnotesize{The application enables users to move photos from
 smart phones, holding personal information, to a shared/public tabletop where
 many people can browse the photos all together}\\ 
 \multicolumn{2}{c}{} \\
 
 & P2P and Service Discovery in Android\\
 &
 \footnotesize{(\url{http://svn.assembla.com/svn/android_p2p/trunk/dk.itu.android.bt.DeviceDiscovery})}\\
 \multicolumn{2}{c}{} \\
 
 \raggedleft \textsc{September 2009} & Kigun \\
 \raggedleft \textsc{April 2010} & \emph{A dental cabinet management project}
 \\ & \footnotesize{The project is built upon the Apache Felix framework (an open source implementation of the OSGi Release 4 core framework specification)
 using the following technologies: Hibernate, Swing, JasperReports} \\ \\
 & \footnotesize{The team has three members (including myself) working from
 different geographical locations, practicing agile development.} \\
 \multicolumn{2}{c}{} \\ 
 
 \raggedleft \emph{March 2009} & Diploma Engineer
 project \\ \raggedleft \textsc{June 2009} & \emph{Automation of Complex Design Flaws Detection} \\
 &\footnotesize{An Eclipse plug-in that detects design problems in projects and gives restructuring hints based on design patterns.}\\
 \multicolumn{2}{c}{} \\
  
 \raggedleft \textsc{2008} & Mini Java compiler \\ 
 &
 \footnotesize{(\url{http://svn2.assembla.com/svn/proiectpt/MiniJavaCompiler})}\\
 & \footnotesize{\href{http://compilers.cs.ucla.edu/vids/MCIIJ2E}{Mini Java} is a subset of Java. The compiler is written Java using the JavaCC parser
 generator.} \\
 \multicolumn{2}{c}{} \\
  
 \raggedleft \textsc{2007} & Download Accelerator \\
 &
 \footnotesize{(\url{http://subversion.assembla.com/svn/old_projects/trunk/DownloadAccelerator})}\\
 & \footnotesize{A client�server application written in standard C, under
 Unix. The application offers the possibility to pause and resume a download. The client can download
 from multiple servers simultaneously.}\\ 
 \multicolumn{2}{c}{} \\[5mm]
 
 & YAGLE\\
 &\footnotesize{(\url{http://subversion.assembla.com/svn/old_projects/trunk/Yagle})}\\
 & \footnotesize{On-line multiplayer arcade games. Written as a Javaclient-server application.}\\
 \multicolumn{2}{c}{} \\
 
 \raggedleft \textsc{2005} & Adolix Wallpaper Changer \\
 &
 \footnotesize{(\url{http://subversion.assembla.com/svn/old_projects/trunk/awc})}\\
 & \footnotesize{A
 desktop wallpaper manager, written in Delphi
 (\href{http://www.adolix.com/wallpaper-changer/}{http://www.adolix.com/wallpaper-changer/}).} \\
 \multicolumn{2}{c}{} \\
 
  \raggedleft \textsc{2003} & File Manager Application \\
  \raggedleft \textsc{2004} &
 \footnotesize{(\url{http://subversion.assembla.com/svn/old_projects/trunk/KNavigator})}\\
 & \footnotesize{Clone of the well known
DosNavigator file manager.} \\
\end{longtable}

\section{Technical Skills}
\begin{tabular}{p{2.5cm}p{11cm}}
\textsc{Programming} & Advanced: C, Java, Pascal \\
\textsc{Scripting} & Intermediate: Delphi, ASM, Tcl/Tk, Bash, PL/SQL \\
& Begginer: C++, Lisp \\
 \multicolumn{2}{c}{} \\ 
 \textsc{Technologies} & XML, Swing, log4j, jUnit, MySQL, Spring, Hibernate,
 GWT, Restlet \\ 
 \multicolumn{2}{c}{} \\
 \textsc{Operating} & MS-DOS, Windows, Linux/UNIX, Android\\
 \textsc{systems}&\\ 
 \multicolumn{2}{c}{} \\
 \textsc{Other} & Knowledge of Eclipse plug-in development environment and model
 driven development technologies (EMF, GMF, XText, XPand)\\
 & Intermediate knowledge of \LaTeX \\
 & HTML, CSS, VHDL\\
\end{tabular}

%Section: Languages
\section{Languages}
\begin{tabular}{p{2.5cm}p{11cm}}
\textsc{Romanian}&Mothertongue\\
\textsc{Hungarian}& Fluent\\
\textsc{English}& Fluent\\
\end{tabular}

\section{Abilities and aptitudes}
Ability to work under pressure.
Creativity, initiative and the ability to work both independently and as part of
a team. Good sense of humour.

\section{Interests and Activities}
I enjoy sports, playing different kind of musical instruments (guitar and
didgeridoo), traveling and culinary arts.

\section{References}
Available upon request.

%\newpage
%\hypertarget{gmat}{\textsc{Gmat}\setmainfont{LMRoman10 Regular}\textregistered\setmainfont[SmallCapsFont=Fontin-SmallCaps]{Fontin-Regular}}

%\XeTeXpdffile ''GMAT.pdf'' page 1 scaled 800

\end{document}
